
% LaTeX file for resume 
% This file uses the resume document class (res.cls)

\documentclass{res} 
%\usepackage{helvetica} % uses helvetica postscript font (download helvetica.sty)  
%\usepackage{newcent}   % uses new century schoolbook postscript font 
\setlength{\textheight}{9.5in} % increase text height to fit on 1-page 

\begin{document} 
\name{GREGORY J. BRUNNER\\[12pt]}     % the \\[12pt] adds a blank
				        % line after name      
\address{\bf HOME ADDRESS\\122 Arabian Path\\ 
St. Peters, MO 63376\\ 636-222-3818\\ gregbrunn@gmail.com}
\address{\bf WORK ADDRESS \\ 3060 Little Hills Expy\\  St. Charles, MO 63301\\  636-949-6620 ext. 8557\\
gbrunner@esri.com}
                                  
\begin{resume}
%\section{OBJECTIVE}          
    %To find a position in a research and development related field that uses my knowledge of physics and allows me to utilize my leadership, collaborative, technical, communication, and writing capabilities
\section{PROFILE}
Mr. Brunner is an experienced Scientist, Programmer, and Professor with a demonstrated history of solving problems involving big data, geographic data, and imagery. He is an expert in the fields of Geographic Information Systems, Geospatial Application Development, Remote Sensing, Big Data Analytics, Python, and ArcGIS. He is an excellent teacher who has taught undergraduate and graduate level geospatial programming courses at St. Louis University. He is an intelligent and passionate research professional with a Master of Science (MS) focused in Physics from Rice University. He has given presentations across the world on topics ranging from Green Infrastructure to Augmented Reality.

\section{EXPERIENCE}
   \vspace{-0.1in}	

   \begin{tabbing}
	\hspace{2.3in}\= \hspace{2.6in}\= \kill % set up two tab positions
	{\bf Senior Scientist} \>Esri     \>April 2016 -- Present\\
	\>St. Charles, MO
\end{tabbing}\vspace{-20pt}      % suppress blank line after tabbing
Currently serves as the lead on multiple projects within Esri. Lead developer on an effort to develop data quality and data comparison algorithms. Lead scientist on an effort to implement large scale raster processing and machine learning algorithms using Amazon Web Services and ArcGIS Image Server. Lead developer for the Esri Python raster function development effort (https://github.com/Esri/raster-functions/). Presented at numerous conferences, including, but not limited to Esri's Federal GIS Conference, Esri's Developer Summit, and Domino Data Lab's Data Science Pop-up.

   \begin{tabbing}
	\hspace{2.3in}\= \hspace{2.6in}\= \kill % set up two tab positions
	{\bf Imagery Scientist} \>Esri     \>July 2011 -- Apr 2016\\
	\>St. Charles, MO
\end{tabbing}\vspace{-20pt}      % suppress blank line after tabbing
Led effort to develop a framework to evaluate large amounts of geographic content. Developed Python tools to automate content assessment. Developed imagery processing pipeline that extracted 3D point clouds from oblique aerial photos. Presented at numerous conferences on topics including, but not limited to 3D augmented reality, imagery analysis, and geographic data quality.

   \begin{tabbing}
	\hspace{2.3in}\= \hspace{2.6in}\= \kill % set up two tab positions
	{\bf Adjunct Professor} \>St. Louis University     \>Jan 2017 -- Present\\
	\>St. Louis, MO
\end{tabbing}\vspace{-20pt}      % suppress blank line after tabbing
Teaches GIS 4091/5091 $-$ $Advanced$ $Programming$ $for$ $GIS$ $and$ $Remote$ $Sensing$ and GIS 4090/5090 $-$ $Programming$ $for$ $Remote$ $Sensing/GIS$. GIS 4090/5090 introduces students to Python programming and its applications to remote sensing and GIS. Through completing this course, students are able to use Python to perform common GIS and remote sensing analysis tasks, automate workflows, and develop custom Python tools. Topics include describing data, manipulating data, automating spatial analysis tasks, creating Python scripts and tools, and using Python for imagery analysis. In GIS 4091/5091, students learn how to publish, consume, and analyze web services using Python, Javascript, and HTML. They are introduced to powerful, advanced Python libraries such as Pandas, Numpy, ArcGIS, and Folium in addition to learning advanced geographic data visualization techniques that leverage Python, Javascript, and web APIs. They also learn how to use the Javascript to create their first stand-alone web applications. This class builds on what students learned in GIS 4090/5090 and helps them develop knowledge and skills that they will use throughout their careers.  


   \begin{tabbing}
	\hspace{2.3in}\= \hspace{2.6in}\= \kill % set up two tab positions
	{\bf GeoSLU Advisory Board} \>St. Louis University     \>Aug 2018 -- Present\\
	\>St. Louis, MO
\end{tabbing}\vspace{-20pt}      % suppress blank line after tabbing
Currently serving on the advisory board for the GeoSLU Big Idea, an initiative to enhance geospatial research, training, and innovation at St. Louis University.

\begin{tabbing}
	\hspace{2.3in}\= \hspace{2.6in}\= \kill % set up two tab positions
	{\bf Geospatial Advisory Board} \>Kaskaskia College     \>Aug 2015 -- May 2018\\
	\>Centralia, IL
\end{tabbing}\vspace{-20pt}      % suppress blank line after tabbing
Served on the Advisory Board for the Kaskaskia College (KC) GIS Program which was awarded a NSF-ATE grant linking a new geospatial technology program at KC to local/regional industry in South-Central IL.
   
   \begin{tabbing}
   \hspace{2.3in}\= \hspace{2.6in}\= \kill % set up two tab positions
    {\bf Associate Scientist} \>Sensing Strategies, Inc.     \>Mar 2008 -- July 2011\\
                             \>Pennington, NJ
   \end{tabbing}\vspace{-20pt}      % suppress blank line after tabbing
Developed algorithms for the analysis of space-based and ground-based sensor data in MATLAB.  Implemented algorithms in data analysis software written in C++.  Designed user interface for viewing sensor data.  Implemented geographic information system software TatukGIS and Google Maps in graphical user interface for analyzing sensor data.  Performed analysis on LiDAR sensor data.  Constructed images from sensor data.  Calibrated sensors in laboratory.  Participated in field tests of remote sensing equipment.

%Helped configure ground stations.  Calibrated sensors used for LASER detection and ranging.  Ran %field tests of remote sensing equipment.
       
   \begin{tabbing}
   \hspace{2.3in}\= \hspace{2.6in}\= \kill % set up two tab positions
    {\bf Graduate Student Researcher} \>Rice University     \>June 2005 -- Mar 2008\\
                             \>Houston, TX
   \end{tabbing}\vspace{-20pt}      % suppress blank line after tabbing
Completed multiple astronomy research projects in collaboration with scientists across the world that led to publications and conference presentations.  Developed computational algorithms to analyze large volumes of spectral data.  Composed a research grant and received approval from Spitzer Science Center and the NASA Jet Propulsion Lab. 
    
      \begin{tabbing}
   \hspace{2.3in}\= \hspace{2.6in}\= \kill % set up two tab positions
    {\bf Visiting Research Fellow} \>Spitzer Science Center, Caltech \> Aug 2006 -- Feb 2007\\
                          \>Pasadena, CA
   \end{tabbing}\vspace{-20pt}
 Worked with $Spitzer$ $Space$ $Telescope$ Infrared Spectrograph instrument team to develop data reduction pipeline that decomposes spectroscopic data cubes into maps.  Wrote paper summarizing our analysis of spectra from nearby galaxies.  
   %\begin{tabbing}%
   %\hspace{2.3in}\= \hspace{2.6in}\= \kill % set up two tab positions          
   %{\bf Independent Researcher}  \>Space Telescope Science Institute \> Sept. 2004 -- Jan. 2005\\
   %                       \>Baltimore, MD
   %\end{tabbing}\vspace{-20pt}
 %Worked with planetary scientists to analyze $Hubble$ $Space$ $Telescope$ imaging and spectroscopic observations of Jupiter's satellite, Europa.       
  
%  \begin{tabbing}%
%   \hspace{2.3in}\= \hspace{2.6in}\= \kill % set up two tab positions          
%  {\bf Research Assistant}  \>Jet Propulsion Laboratory, NASA \> ~~~~~~Summer 2004\\
%                         \>Pasadena, CA
%   \end{tabbing}\vspace{-20pt}
%    Helped set up a lab to study the superfluid properties of liquid helium.  
%    Created hardware for superfluid liquid helium experiment and software that was used to analyze the results. 

\section{EDUCATION}
    Rice University, Houston, TX \\
    M.S., Physics, 2008
    
    Johns Hopkins University, Baltimore, MD  \\        
    B.S., Physics, 2005\\
    Minor in Earth and Planetary Science       
    
\section{HONORS AND AWARDS} 
	Certificate of Appreciation from the National Geospatial-Intelligence Agency's Geosaptial Analyst Hub (GA Hub) for $Capturing$ $and$ $Sharing$ $Technical$ $How-To$, June 2018\\
	Certificate of Appreciation from the National Geospatial-Intelligence Agency's Geosaptial Analyst Hub (GA Hub) for $Excellence$ $in$ $Using$ $Statistics$ $in$ $Python$, July 2017\\
	Certificate of Appreciation from the National Geospatial-Intelligence Agency's Geosaptial Analyst Hub (GA Hub) for $Python$ $Party$ $Presenting$ $-$ $Sharing$ $Scripting$ $Expertise$ $with$ $Others$ $in$ $an$ $Engaging$ $Manner$, June 2016\\
   Spitzer Visiting Graduate Student Fellow, Spitzer Science Center, 2006--2007   \\      
   NASA Undergraduate Student Research Program Fellow, NASA Jet Propulsion Lab, Summer 2004\\        
   Caltech Summer Undergraduate Research Fellow, California Institute of Technology, Summer 2004  

\section{PROFESSIONAL AFFILIATIONS}      
The American Society for Photogrammetry and Remote Sensing, ASPRS (2011 - 2018)\\
Association for Unmanned Vehicle Systems International, AUVSI (2014 - 2016)\\
ASPRS National Board of Directors (2015 - 2016)\\
ASPRS Heartland Region Director (2015 - 2016)\\
ASPRS Heartland Region President (2013 - 2014)\\
The American Astronomical Society (2006 - 2011)\\


\section{STATEMENT OF TEACHING PHILOSOPHY}
https://cei.umn.edu/writing-your-teaching-philosophy
 
 
\section{STUDENT COMMENTS FROM COURSE EVALUATIONS}
   \vspace{-0.1in}	

\begin{tabbing}
	\hspace{2.3in}\= \hspace{2.6in}\= \kill % set up two tab positions
	{\bf GIS 4090/5090} \>Programming for Remote Sensing-GIS     \>Spring 2018\\
	\>
\end{tabbing}\vspace{-20pt}      % suppress blank line after tabbing
``Greg is really fantastic! He was so helpful and patient with all of the students, no matter our experience level. I honestly have no complaints. Great class, wonderful instructor. I'd take classes from him 1000 times over."\\
``I feel that Greg was enthusiastic about teaching us how to write code and was encouraging if we ever ran into issues. I feel like he had good knowledge to provide to us since he works at ESRI and is not only in the academic field."\\
``Great teacher, always there to help"\\
``Greg was great about answering emails and being available outside of normal class hours for help."\\
``He was a fantastic instructor and a very patient man. I would suggest him for another course"\\
``Greg was very respectful, insightful, enthusiastic."\\
``He is an energetic and happy person. Thanks to Gregory for giving us such a wonderful course."\\
``very knowledgeable on the subject and taught in a way that was easy to understand."\\
``Very accommodating and he knows the programming very well!"\\
``I thought the course was great and exactly what I was looking for in a programming Python course."\\
``Great class, no need to change"\\
``...it was a fantastic course to enable programmers to use ArcGIS more effectively"\\
``I really enjoyed this course! It was challenging to code for the first time, but the material was presented well. The projects were difficult but doable and really forced you to think. Some homework assignments required just repeating information from the book, but I can't think of a different way to learn some of the material. Overall, it was a great class. I wish that I had more time to spend with the material, but that is no fault of the instructor or the course."\\
``This was a very well planned course. It was paced well and was engaging throughout."\\
``Great course–– I was a bit intimidated at first but it went at a nice pace that I could handle."\\
``I enjoyed this course, and like that I am able to apply the knowledge I gained to my current job and career. I feel that it had a good balance of assignments and projects."\\

\begin{tabbing}
	\hspace{2.3in}\= \hspace{2.6in}\= \kill % set up two tab positions
	{\bf GIS 4091/5091} \>Advanced Programming for GIS     \>Fall 2018\\
	\>
\end{tabbing}\vspace{-20pt}      % suppress blank line after tabbing
TBD

\end{resume}
\end{document}