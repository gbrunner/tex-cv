
% LaTeX file for resume 
% This file uses the resume document class (res.cls)

\documentclass{res} 
%\usepackage{helvetica} % uses helvetica postscript font (download helvetica.sty)  
%\usepackage{newcent}   % uses new century schoolbook postscript font 
\setlength{\textheight}{9.5in} % increase text height to fit on 1-page 


\begin{document} 
\name{GREGORY J. BRUNNER\\[12pt]}     % the \\[12pt] adds a blank
				        % line after name      
\address{\bf HOME ADDRESS\\122 Arabian Path\\ 
St. Peters, MO 63376\\ 636-222-3818\\ gregbrunn@gmail.com}
\address{\bf WORK ADDRESS \\ 3060 Little Hills Expy\\  St. Charles, MO 63301\\  636-949-6620 ext. 8557\\
gbrunner@esri.com}
                                  
\begin{resume}
%\section{OBJECTIVE}          
    %To find a position in a research and development related field that uses my knowledge of physics and allows me to utilize my leadership, collaborative, technical, communication, and writing capabilities
\section{PROFILE}
Mr. Brunner is an experienced Scientist, Programmer, and Professor with a demonstrated history of solving problems involving big data, geographic data, and imagery. He is an expert in the fields of Geographic Information Systems, Geospatial Application Development, Remote Sensing, Big Data Analytics, Python, and ArcGIS. He is an excellent teacher who has taught undergraduate and graduate level geospatial programming courses at St. Louis University. He is an intelligent and passionate research professional with a Master of Science (MS) in Physics from Rice University. He has given presentations across the world on topics ranging from Green Infrastructure to Augmented Reality.

\section{EXPERIENCE}
   \vspace{-0.1in}	

   \begin{tabbing}
	\hspace{2.3in}\= \hspace{2.6in}\= \kill % set up two tab positions
	{\bf Senior Scientist} \>Esri     \>April 2016 -- Present\\
	\>St. Charles, MO
\end{tabbing}\vspace{-20pt}      % suppress blank line after tabbing
Currently serves as the lead on multiple projects within Esri. Lead developer on an effort to develop data quality and data comparison algorithms. Lead scientist on an effort to implement large scale raster processing and machine learning algorithms using Amazon Web Services and ArcGIS Image Server. Lead developer for the Esri Python raster function development effort (https://github.com/Esri/raster-functions/). Presented at numerous conferences, including but not limited to Esri's Federal GIS Conference, Esri's Developer Summit, and Domino Data Lab's Data Science Pop-up.

   \begin{tabbing}
	\hspace{2.3in}\= \hspace{2.6in}\= \kill % set up two tab positions
	{\bf Imagery Scientist} \>Esri     \>July 2011 -- Apr 2016\\
	\>St. Charles, MO
\end{tabbing}\vspace{-20pt}      % suppress blank line after tabbing
Led effort to develop a framework to evaluate large amounts of geographic content. Developed Python tools to automate content assessment. Developed imagery processing pipeline that extracted 3D point clouds from oblique aerial photos. Presented at numerous conferences on topics including but not limited to 3D augmented reality, imagery analysis, and geographic data quality.

   \begin{tabbing}
	\hspace{2.3in}\= \hspace{2.6in}\= \kill % set up two tab positions
	{\bf Adjunct Professor} \>St. Louis University     \>Jan 2017 -- Present\\
	\>St. Louis, MO
\end{tabbing}\vspace{-20pt}      % suppress blank line after tabbing
Teaches GIS 4090/5090 - \textit{Introduction to Programming for GIS and Remote Sensing} and GIS 4091/5091 - \textit{Advanced Programming for GIS}. GIS 4090/5090 introduces students to Python programming and its applications to remote sensing and GIS. Through completing this course, students are able to use Python to perform common GIS and remote sensing analysis tasks, automate workflows, and develop custom Python tools. In GIS 4091/5091, students learn how to publish, consume, and analyze web services using Python, Javascript, and HTML. They are introduced to powerful, advanced Python libraries such as Pandas, Numpy, ArcGIS, and Folium in addition to learning advanced geographic data visualization techniques that leverage Python, Javascript, and web APIs. They also learn how to use Javascript to create their first stand-alone web applications. 
%Topics include describing data, manipulating data, automating spatial analysis %tasks, creating Python scripts and tools, and using Python for imagery analysis. 

%This class builds on what students learned in GIS 4090/5090 and helps them develop knowledge and skills that they will use throughout their careers.  


   \begin{tabbing}
	\hspace{2.3in}\= \hspace{2.6in}\= \kill % set up two tab positions
	{\bf GeoSLU Advisory Board} \>St. Louis University     \>Aug 2018 -- Present\\
	\>St. Louis, MO
\end{tabbing}\vspace{-20pt}      % suppress blank line after tabbing
Currently serving on the advisory board for the GeoSLU Big Idea, an initiative to enhance geospatial research, training, and innovation at St. Louis University.

\begin{tabbing}
	\hspace{2.3in}\= \hspace{2.6in}\= \kill % set up two tab positions
	{\bf Geospatial Advisory Board} \>Kaskaskia College     \>Aug 2015 -- May 2018\\
	\>Centralia, IL
\end{tabbing}\vspace{-20pt}      % suppress blank line after tabbing
Served on the Advisory Board for the Kaskaskia College (KC) GIS Program which was awarded a NSF-ATE grant linking a new geospatial technology program at KC to local/regional industry in South-Central Illinois.
   
   \begin{tabbing}
   \hspace{2.3in}\= \hspace{2.6in}\= \kill % set up two tab positions
    {\bf Associate Scientist} \>Sensing Strategies, Inc.     \>Mar 2008 -- July 2011\\
                             \>Pennington, NJ
   \end{tabbing}\vspace{-20pt}      % suppress blank line after tabbing
Developed algorithms for the analysis of space-based and ground-based sensor data in MATLAB.  Implemented algorithms in data analysis software written in C++.  Designed user interface for viewing sensor data.  Implemented geographic information system software TatukGIS and Google Maps in graphical user interface for analyzing sensor data.  Performed analysis on LiDAR sensor data.  Constructed images from sensor data.  Calibrated sensors in laboratory.  Participated in field tests of remote sensing equipment.

%Helped configure ground stations.  Calibrated sensors used for LASER detection and ranging.  Ran %field tests of remote sensing equipment.
       
   \begin{tabbing}
   \hspace{2.3in}\= \hspace{2.6in}\= \kill % set up two tab positions
    {\bf Graduate Student Researcher} \>Rice University     \>June 2005 -- Mar 2008\\
                             \>Houston, TX
   \end{tabbing}\vspace{-20pt}      % suppress blank line after tabbing
Completed multiple astronomy research projects in collaboration with scientists across the world that led to publications and conference presentations.  Developed computational algorithms to analyze large volumes of spectral data.  Composed a research grant and received approval from Spitzer Science Center and the NASA Jet Propulsion Lab. 
    
      \begin{tabbing}
   \hspace{2.3in}\= \hspace{2.6in}\= \kill % set up two tab positions
    {\bf Visiting Research Fellow} \>Spitzer Science Center, Caltech \> Aug 2006 -- Feb 2007\\
                          \>Pasadena, CA
   \end{tabbing}\vspace{-20pt}
 Worked with $Spitzer$ $Space$ $Telescope$ Infrared Spectrograph instrument team to develop data reduction pipeline that decomposes spectroscopic data cubes into maps.  Wrote paper summarizing analysis of spectra from nearby galaxies.  
   %\begin{tabbing}%
   %\hspace{2.3in}\= \hspace{2.6in}\= \kill % set up two tab positions          
   %{\bf Independent Researcher}  \>Space Telescope Science Institute \> Sept. 2004 -- Jan. 2005\\
   %                       \>Baltimore, MD
   %\end{tabbing}\vspace{-20pt}
 %Worked with planetary scientists to analyze $Hubble$ $Space$ $Telescope$ imaging and spectroscopic observations of Jupiter's satellite, Europa.       
  
%  \begin{tabbing}%
%   \hspace{2.3in}\= \hspace{2.6in}\= \kill % set up two tab positions          
%  {\bf Research Assistant}  \>Jet Propulsion Laboratory, NASA \> ~~~~~~Summer 2004\\
%                         \>Pasadena, CA
%   \end{tabbing}\vspace{-20pt}
%    Helped set up a lab to study the superfluid properties of liquid helium.  
%    Created hardware for superfluid liquid helium experiment and software that was used to analyze the results. 

\section{EDUCATION}
    Rice University, Houston, TX \\
    M.S., Physics, 2008
    
    Johns Hopkins University, Baltimore, MD  \\        
    B.S., Physics, 2005\\
    Minor in Earth and Planetary Science       
    
\section{HONORS AND AWARDS} 
	Certificate of Appreciation from the National Geospatial-Intelligence Agency's Geospatial Analyst Hub (GA Hub) for $Capturing$ $and$ $Sharing$ $Technical$ $How-To$, June 2018\\
	Certificate of Appreciation from the National Geospatial-Intelligence Agency's Geospatial Analyst Hub (GA Hub) for $Excellence$ $in$ $Using$ $Statistics$ $in$ $Python$, July 2017\\
	Certificate of Appreciation from the National Geospatial-Intelligence Agency's Geospatial Analyst Hub (GA Hub) for $Python$ $Party$ $Presenting$ $-$ $Sharing$ $Scripting$ $Expertise$ $with$ $Others$ $in$ $an$ $Engaging$ $Manner$, June 2016\\
   Spitzer Visiting Graduate Student Fellow, Spitzer Science Center, 2006--2007   \\      
   NASA Undergraduate Student Research Program Fellow, NASA Jet Propulsion Lab, Summer 2004\\        
   Caltech Summer Undergraduate Research Fellow, California Institute of Technology, Summer 2004  

\section{PROFESSIONAL AFFILIATIONS}      
The American Society for Photogrammetry and Remote Sensing, ASPRS (2011 - 2018)\\
Association for Unmanned Vehicle Systems International, AUVSI (2014 - 2016)\\
ASPRS National Board of Directors (2015 - 2016)\\
ASPRS Heartland Region Director (2015 - 2016)\\
ASPRS Heartland Region President (2013 - 2014)\\
The American Astronomical Society (2006 - 2011)\\


\section{STATEMENT OF TEACHING PHILOSOPHY}
%\setlength{\parindent}{1cm} % Default is 15pt.
%\noindent
%\newline
My goal is to position my students to get their dream job. The graduate students that take my class are there to learn new programming, analytical, and mapping skills to prepare themselves for their first job or launch themselves into a new one. The undergraduates are there to prepare themselves for graduate school or their first post-collegiate job. All of my students have identified that learning how to program is a skill that will help them achieve their career goals and all of my students have an interest in diving deeper into creating maps and using geographic information systems (GIS) to solve problems of local, regional, and global importance.

\par The past two years as an Adjunct Instructor in the Department of Earth and Atmospheric Sciences, I have taught both \textit{Introduction to Programming for GIS and Remote Sensing} (GIS 4090/5090) and \textit{Advanced Programming for GIS} (GIS 4091/5091). These courses are cross listed for both undergraduate and graduate students to take. Of the graduate students who take my class, some are full-time graduate students pursuing a master’s degree or a PhD and some are part-time students who also work full-time jobs. The students who enroll in these classes come from diverse fields of study, such as meteorology, biology, geology, public health, and social science. I strive to prepare each of my students for a career in their chosen discipline. It can be challenging to reach each student in a class with such diverse interests. My classes can be challenging for students who want to learn how to program but are intimidated by the topic. It is my job to overcome these challenges and I do this in several ways:
\newline
%\par This goal is shaped by my experience as a professional Scientist in the field of Geographic Information Science. The most valuable skills I have are the knowledge to solve analytical problems involving geography, the ability to implement solutions by programming or writing code, and the confidence to clearly communicate the solution that I have developed. The combination of those skills have opened up opportunities throughout my career and it is those skills that I try to instill in my students. 
\begin{enumerate}
\item I mentor. I put tremendous effort into creating the materials I use to teach, but I am also eager to talk to students about applications of what they are learning and potential career paths that are open to them. 
\item I make myself very available. During class periods, I spend about half the time lecturing, discussing, and demonstrating techniques. I use the other half for the students to go through hands on programming exercises. I encourage them to ask their classmates and me for help when they run into problems or confusion. If they are unable to complete everything during class time, I am reachable via email and almost always respond within the day.
\item I tailor the class to different skill levels. I have a baseline set of classwork and homework exercises that I want all students to be able to complete but I also offer optional additional problems for homework for students who are looking for a greater challenge.
\item I use a lot of real-world examples in lectures and assignments that cover the varying disciplines of my students. Students learn how they can use programming to create maps of refugee movement based on data from the United Nations High Commission for Refugees, they apply their knowledge and skills to create an earthquake area of impact estimator based on the USGS Earthquake Pager, and they create crime trend maps based on data provided by the St. Louis Metro Police Department.
\item I give students the freedom to define their own projects. For several projects, I give students the ability to choose between defining their own project or taking on a project that I have defined. This approach has been very fruitful for students because each student can tailor projects to their own interests. The results have been impressive. For example:
\begin{itemize}  
\item Emma Blackwood mapped vacant lots and food desert tracts in STL - 
https://bit.ly/2AiV9D7 
\item Wayne Moore mapped Ebola cases in West Africa - https://bit.ly/2CAfSE2 
\item Lauren Lovato created a 3D map of Mars that includes popups of 3D anaglyph images - https://bit.ly/2ESZKj7
\end{itemize}
\end{enumerate}
\par When I see beautiful maps, webpages, and results such as the ones listed, I know that my approach to teaching is effective and that my students will be well-prepared for a career in their chosen discipline. Seeing tangible results and seeing the enthusiasm of the students as they work on these projects is what motivates me to teach at St. Louis University. In my career, I have had some fantastic teachers and mentors. The knowledge and skills they have instilled in me have put me in a position to be a mentor to others. It is my privilege and my goal to provide the education and mentorship to my students to prepare them for their dream job.
 
 
\section{STUDENT COMMENTS FROM COURSE EVALUATIONS}
   \vspace{-0.1in}	

\begin{tabbing}
	\hspace{2.3in}\= \hspace{2.6in}\= \kill % set up two tab positions
	{\bf GIS 4090/5090} \>Programming for Remote Sensing-GIS     \>Spring 2018\\
	\>
\end{tabbing}\vspace{-20pt}      % suppress blank line after tabbing
\begin{itemize}  
\item``Greg is really fantastic! He was so helpful and patient with all of the students, no matter our experience level. I honestly have no complaints. Great class, wonderful instructor. I'd take classes from him 1000 times over."
\item``I feel that Greg was enthusiastic about teaching us how to write code and was encouraging if we ever ran into issues. I feel like he had good knowledge to provide to us since he works at ESRI and is not only in the academic field."
\item``Great teacher, always there to help"
\item``Greg was great about answering emails and being available outside of normal class hours for help."
\item``He was a fantastic instructor and a very patient man. I would suggest him for another course"
\item``Greg was very respectful, insightful, enthusiastic."
\item``He is an energetic and happy person. Thanks to Gregory for giving us such a wonderful course."
\item``very knowledgeable on the subject and taught in a way that was easy to understand."
\item``Very accommodating and he knows the programming very well!"
\item``I thought the course was great and exactly what I was looking for in a programming Python course."
\item``Great class, no need to change"
\item``...it was a fantastic course to enable programmers to use ArcGIS more effectively"
\item``I really enjoyed this course! It was challenging to code for the first time, but the material was presented well. The projects were difficult but doable and really forced you to think. Some homework assignments required just repeating information from the book, but I can't think of a different way to learn some of the material. Overall, it was a great class. I wish that I had more time to spend with the material, but that is no fault of the instructor or the course."
\item``This was a very well planned course. It was paced well and was engaging throughout."
\item``Great course–– I was a bit intimidated at first but it went at a nice pace that I could handle."
\item``I enjoyed this course, and like that I am able to apply the knowledge I gained to my current job and career. I feel that it had a good balance of assignments and projects."
\end{itemize}
\begin{tabbing}
	\hspace{2.3in}\= \hspace{2.6in}\= \kill % set up two tab positions
	{\bf GIS 4091/5091} \>Advanced Programming for GIS     \>Fall 2018\\
	\>
\end{tabbing}\vspace{-20pt}      % suppress blank line after tabbing
\begin{itemize}  
\item ``Course was fun, the content was interesting. Would highly recommend this course for individuals who are serious about advancing their GIS developing skills."
\item ``I thought that this class was very interesting and useful, and liked that we were able to learn more topics than what was covered in the intro class. Learning the web–based stuff was fun and helpful, it seems like a good option if presenting any kind of data or maps. I hope to use what I learned in the future! I really liked this class!"
\item ``Love the Tegrity recorded lectures and I'm glad to have been exposed to Jupyter notebooks."
\item ``It was very clear the instructor enjoys the subject and encouraging students to become enthusiastic too."
\item ``He always seemed excited to teach us new and cool things"
\item ``Instructor always provided feedback for both informing us of good work or if there could be improvement"
\item ``Always took time to answer our questions and listen to any concerns/issues we had"
\item ``Instructor did his best to respond to students in a timely manner."
\item ``This was a good class. Thank you for the semester. It was fun!"
\item ``Awesome class, thanks Greg!"
\end{itemize}
\end{resume}

%\section{REFERENCES}
%\vspace{-0.1in}	

%\begin{thebibliography}{9}
%\bibitem{unhcr} 
%\textit{The United Nations High Commission for Refugees}. 
%https://www.unhcr.org/
%\end{thebibliography}

\end{document}