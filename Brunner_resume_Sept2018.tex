
% LaTeX file for resume 
% This file uses the resume document class (res.cls)

\documentclass{res} 
%\usepackage{helvetica} % uses helvetica postscript font (download helvetica.sty)  
%\usepackage{newcent}   % uses new century schoolbook postscript font 
\setlength{\textheight}{9.5in} % increase text height to fit on 1-page 

\begin{document} 
\name{GREGORY J. BRUNNER\\[12pt]}     % the \\[12pt] adds a blank
				        % line after name      
\address{\bf HOME ADDRESS\\122 Arabian Path\\ 
St. Peters, MO 63376\\ 636-222-3818\\ gregbrunn@gmail.com}
\address{\bf \hfill PROFILE \\ \hfill linkedin.com/in/gregbrunn
	\\ \hfill github.com/gbrunner
	\\ \hfill @gregbrunn}
%\address{\bf WORK ADDRESS \\ 3060 Little Hills Expy\\  St. Charles, MO 63301\\  636-949-6620 ext. 8557\\
%gbrunner@esri.com}
                                  
\begin{resume}
%\section{OBJECTIVE}          
    %To find a position in a research and development related field that uses my knowledge of physics and allows me to utilize my leadership, collaborative, technical, communication, and writing capabilities
\section{PROFILE}
Mr. Brunner is an experienced Scientist, Programmer, and Professor with a demonstrated history of solving problems involving big data, geographic data, and imagery. He is an expert in the fields of Geographic Information Systems, Geospatial Application Development, Remote Sensing, Big Data Analytics, Python, and ArcGIS. He is an excellent teacher who has taught undergraduate and graduate level geospatial programming courses at St. Louis University. He is an intelligent and passionate research professional with a Master of Science (MS) focused in Physics from Rice University. He has given presentations across the world on topics ranging from Green Infrastructure to Augmented Reality.

\section{EXPERIENCE}
   \vspace{-0.1in}	

   \begin{tabbing}
	\hspace{2.3in}\= \hspace{2.6in}\= \kill % set up two tab positions
	{\bf Senior Scientist} \>Esri     \>April 2016 -- Present\\
	\>St. Charles, MO
\end{tabbing}\vspace{-20pt}      % suppress blank line after tabbing
Currently serves as the lead on multiple projects within Esri. Lead developer on an effort to develop data quality and data comparison algorithms. Lead scientist on an effort to implement large scale raster processing and machine learning algorithms using Amazon Web Services and ArcGIS Image Server. Lead developer for the Esri Python raster function development effort (https://github.com/Esri/raster-functions/). Presented at numerous conferences, including, but not limited to Esri's Federal GIS Conference, Esri's Developer Summit, and Domino Data Lab's Data Science Pop-up.

   \begin{tabbing}
	\hspace{2.3in}\= \hspace{2.6in}\= \kill % set up two tab positions
	{\bf Imagery Scientist} \>Esri     \>July 2011 -- Apr 2016\\
	\>St. Charles, MO
\end{tabbing}\vspace{-20pt}      % suppress blank line after tabbing
Led effort to develop a framework to evaluate large amounts of geographic content. Developed Python tools to automate content assessment. Developed imagery processing pipeline that extracted 3D point clouds from oblique aerial photos. Presented at numerous conferences on topics including, but not limited to 3D augmented reality, imagery analysis, and geographic data quality.

   \begin{tabbing}
	\hspace{2.3in}\= \hspace{2.6in}\= \kill % set up two tab positions
	{\bf Adjunct Professor} \>Saint Louis University     \>Jan 2017 -- Present\\
	\>St. Louis, MO
\end{tabbing}\vspace{-20pt}      % suppress blank line after tabbing
Teaches GIS 4091/5091 $-$ $Advanced$ $Programming$ $for$ $GIS$ $and$ $Remote$ $Sensing$ and GIS 4090/5090 $-$ $Programming$ $for$ $Remote$ $Sensing/GIS$. GIS 4090/5090 introduces students to Python programming and its applications to remote sensing and GIS. Through completing this course, students are able to use Python to perform common GIS and remote sensing analysis tasks, automate workflows, and develop custom Python tools. Topics include describing data, manipulating data, automating spatial analysis tasks, creating Python scripts and tools, and using Python for imagery analysis. In GIS 4091/5091, students learn how to publish, consume, and analyze web services using Python, Javascript, and HTML. They are introduced to powerful, advanced Python libraries such as Pandas, Numpy, ArcGIS, and Folium in addition to learning advanced geographic data visualization techniques that leverage Python, Javascript, and web APIs. They also learn how to use the Javascript to create their first stand-alone web applications. This class builds on what students learned in GIS 4090/5090 and helps them develop knowledge and skills that they will use throughout their careers.  


   \begin{tabbing}
	\hspace{2.3in}\= \hspace{2.6in}\= \kill % set up two tab positions
	{\bf GeoSLU Advisory Board} \>Saint Louis University     \>Aug 2018 -- Present\\
	\>St. Louis, MO
\end{tabbing}\vspace{-20pt}      % suppress blank line after tabbing
Currently serving on the advisory board for the GeoSLU Big Idea, an initiative to enhance geospatial research, training, and innovation at Saint Louis University.

%\begin{tabbing}
%	\hspace{2.3in}\= \hspace{2.6in}\= \kill % set up two tab positions
%	{\bf Geospatial Advisory Board} \>Kaskaskia College     \>Aug 2015 -- %May 2018\\
%	\>Centralia, IL
%\end{tabbing}\vspace{-20pt}      % suppress blank line after tabbing
%Served on the Advisory Board for the Kaskaskia College (KC) GIS Program %which was awarded a NSF-ATE grant linking a new geospatial technology %program at KC to local/regional industry in South-Central IL.
   
   \begin{tabbing}
   \hspace{2.3in}\= \hspace{2.6in}\= \kill % set up two tab positions
    {\bf Associate Scientist} \>Sensing Strategies, Inc.     \>Mar 2008 -- July 2011\\
                             \>Pennington, NJ
   \end{tabbing}\vspace{-20pt}      % suppress blank line after tabbing
Developed algorithms for the analysis of space-based and ground-based sensor data in MATLAB.  Implemented algorithms in data analysis software written in C++.  Designed user interface for viewing sensor data.  Implemented geographic information system software TatukGIS and Google Maps in graphical user interface for analyzing sensor data.  Performed analysis on LiDAR sensor data.  Constructed images from sensor data.  Calibrated sensors in laboratory.  Participated in field tests of remote sensing equipment.

%Helped configure ground stations.  Calibrated sensors used for LASER detection and ranging.  Ran %field tests of remote sensing equipment.
       
   \begin{tabbing}
   \hspace{2.3in}\= \hspace{2.6in}\= \kill % set up two tab positions
    {\bf Graduate Student Researcher} \>Rice University     \>June 2005 -- Mar 2008\\
                             \>Houston, TX
   \end{tabbing}\vspace{-20pt}      % suppress blank line after tabbing
Completed multiple astronomy research projects in collaboration with scientists across the world that led to publications and conference presentations.  Developed computational algorithms to analyze large volumes of spectral data.  Composed a research grant and received approval from Spitzer Science Center and the NASA Jet Propulsion Lab. 
    
      \begin{tabbing}
   \hspace{2.3in}\= \hspace{2.6in}\= \kill % set up two tab positions
    {\bf Visiting Research Fellow} \>Spitzer Science Center, Caltech \> Aug 2006 -- Feb 2007\\
                          \>Pasadena, CA
   \end{tabbing}\vspace{-20pt}
 Worked with $Spitzer$ $Space$ $Telescope$ Infrared Spectrograph instrument team to develop data reduction pipeline that decomposes spectroscopic data cubes into maps.  Wrote paper summarizing our analysis of spectra from nearby galaxies.  
   %\begin{tabbing}%
   %\hspace{2.3in}\= \hspace{2.6in}\= \kill % set up two tab positions          
   %{\bf Independent Researcher}  \>Space Telescope Science Institute \> Sept. 2004 -- Jan. 2005\\
   %                       \>Baltimore, MD
   %\end{tabbing}\vspace{-20pt}
 %Worked with planetary scientists to analyze $Hubble$ $Space$ $Telescope$ imaging and spectroscopic observations of Jupiter's satellite, Europa.       
  
%  \begin{tabbing}%
%   \hspace{2.3in}\= \hspace{2.6in}\= \kill % set up two tab positions          
%  {\bf Research Assistant}  \>Jet Propulsion Laboratory, NASA \> ~~~~~~Summer 2004\\
%                         \>Pasadena, CA
%   \end{tabbing}\vspace{-20pt}
%    Helped set up a lab to study the superfluid properties of liquid helium.  
%    Created hardware for superfluid liquid helium experiment and software that was used to analyze the results. 
    



\section{EDUCATION}
    Rice University, Houston, TX \\
    M.S., Physics, 2008
    
    Johns Hopkins University, Baltimore, MD  \\        
    B.S., Physics, 2005\\
    Minor in Earth and Planetary Science       
    
\section{HONORS AND AWARDS} 
	2019 Excellence in Adjunct Teaching Award from Saint Louis University\\	
	Certificate of Appreciation from the National Geospatial-Intelligence Agency's Geosaptial Analyst Hub (GA Hub) for $Capturing$ $and$ $Sharing$ $Technical$ $How-To$, June 2018\\
	Certificate of Appreciation from the National Geospatial-Intelligence Agency's Geosaptial Analyst Hub (GA Hub) for $Excellence$ $in$ $Using$ $Statistics$ $in$ $Python$, July 2017\\
	Certificate of Appreciation from the National Geospatial-Intelligence Agency's Geosaptial Analyst Hub (GA Hub) for $Python$ $Party$ $Presenting$ $-$ $Sharing$ $Scripting$ $Expertise$ $with$ $Others$ $in$ $an$ $Engaging$ $Manner$, June 2016\\
   Spitzer Visiting Graduate Student Fellow, Spitzer Science Center, 2006--2007   \\      
   NASA Undergraduate Student Research Program Fellow, NASA Jet Propulsion Lab, Summer 2004\\        
   Caltech Summer Undergraduate Research Fellow, California Institute of Technology, Summer 2004  

\section{PROFESSIONAL AFFILIATIONS}      
The American Society for Photogrammetry and Remote Sensing, ASPRS (2011 - 2018)\\
Association for Unmanned Vehicle Systems International, AUVSI (2014 - 2016)\\
ASPRS Heartland Region Director (2015 - 2016)\\
ASPRS Heartland Region President (2013 - 2014)\\
The American Astronomical Society (2006 - 2011)\\


%\section{COMPUTER SKILLS}          
%    Extensive knowledge of software for Windows, Mac OS X, UNIX, and Linux. \\         
%    Proficient programming skills in C, C++, Python, MATLAB, and IDL.\\
%    Expert with commercial GIS software such as Esri ArcGIS and TatukGIS.\\
    %Familiarity with open source   
    %Ability to program in Objective-C using Cocoa on Mac OS X.\\
    %Familiar with commercial GIS software such as ESRI products and TatukGIS.\\
    %Experience with OpenGL and Windows APIs. \\
   % Proficient with image processing and analysis software packages such as IRAF and GIMP.           

%\section{RESEARCH GRANTS}
%Spitzer Archival Research Program \#40773: \textit{Mapping Molecular Hydrogen Excitation and Mass in Nearby Galaxies from the SINGS Archive.}  PI: Brunner, G. %Amount awarded: \$96,765.  (For more info, see http://ssc.spitzer.caltech.edu/geninfo/ar/abs-ar4/40773.txt)
 
 
\section{SELECTED PRESENTATIONS}

\textit{GeoPlanner for ArcGIS: Use in Green Infrastructure Planning and Beyond.} Esri Federal GIS Conference. Washington, DC. March 2018.

\textit{Making Data Science Tanglible} Domino Data Pop-up. Chicago, IL. November 2017.

\textit{Enabling Apps for Green Infrastructure Planning.} Esri Federal GIS Conference. Washington, DC. February 2017.

\textit{Thunder Up! Mapping Russell Westbrook's Shots with Python and ArcGIS.} Oklahoma South Central ArcGIS User Group.  Oklahoma City, OK.  September 2016.

\textit{Global Data Management Using Content Suitability Indicators.} 2016 Esri International User Conference.  San Diego, CA.  July 2016.

\textit{Writing Image Processing Algorithms with the Python Raster Function.} 2016 Esri Developer Summit.  Palm Springs, CA.  March 2016.

\textit{What's the State of the Data?} 2015 Esri International User Conference.  San Diego, CA.  July 2015.

\textit{GeoPlanner for ArcGIS: An Overview.} 2015 Esri Federal User Conference.  Washington, DC.  February 2015.

\textit{Not a Web-Developer? Not a Problem! Using Open, Free, Configurable GIS Web-Applications to 
Jumpstart Your GIS Application Development Project.} ASPRS 2014 Annual Conference.  Thought Leader Presentation. Louisville, KY.  March 2014.

\textit{Managing Airport Airspace Using 3D Augmented Reality.} Air Traffic Controllers Association International Conference.  Amsterdam, The Netherlands.  March 2013.

\textit{Graduate Education and Careers in Science.} Holy Ghost Preparatory School Senior Career Day.  Bensalem, PA.  October 2008, 2009, and 2010.

\textit{Mapping the Spatial Distribution of H$_2$ and PAH Emission in Nearby Galaxies with the Spitzer IRS.}  211$^{th}$ meeting of the American Astronomical Society.  Austin, TX.  January 2008.

%\textit{Observations of M33 HII regions: The Ne/S Ratio, metallicity, and ionization variations.} Rubin, B., Simpson, J., McNabb, I., Brunner, G., Colgan, S., Dufour, R., and Pauldrach, A.  Spitzer Conference on the ISM in the Milky Way and Nearby Galaxies. Pasadena, CA.  December, 2007.

\textit{Mapping the Spatial Distribution of Warm H$_2$ in Nearby Galaxies with the Spitzer IRS.}  Spitzer Conference on the ISM in the Milky Way and Nearby Galaxies.  Pasadena, CA.  December 2007.

%\textit{Mapping PAHs in Nearby Galaxies with Spitzer.}  Astronomical Units Seminar. Rice University.  November, 2007.

%\textit{Astronomy in 3D: Spectroscopy and Modeling of Nebulae.}  Astronomical Units Seminar. Rice University, September 2007.

%\textit{Warm Molecular Gas in M51: Mapping the Excitation Temperature and Mass of H$_2$ with the Spitzer Infrared Spectrograph.}  Master�s Thesis Defense.  Rice University.  September 17, 2007.

\textit{Mapping Ne, S, and H$_2$ across M51 with the Spitzer IRS.}  Deep Spectroscopy and Modeling  of Nebular Emission Workshop. Beijing, China.  April 2007.

%\textit{Exploring Galaxies in the Infrared with the Spitzer Space Telescope.} University of Houston, Houston, TX. Invited speaker at the Houston Astronomical Society Meeting. March, 2007.

%\textit{Mapping H$_2$ Excitation and Mass across M51 with the Spitzer IRS.} Rice University Astronomical Unit Seminar.  Rice University, Houston, TX.  February, 2007.

%\textit{Mapping H$_2$ Temperature, Mass, and Ortho-to-Para Ratio across NGC 5194.} Spitzer Science Center Lunch Seminar Series. Spitzer Science Center, Pasadena, CA.  January, 2007.

\textit{Spitzer IRS Spectral Maps of Spatially Resolved Molecular Hydrogen in NGC5194.}  209$^{th}$ Meeting of the American Astronomical Society.  Seattle, WA.  January 2007.

%\textit{Mapping H$_2$ in M51 with the Spitzer IRS.} Greater IPAC Science Symposium. Caltech, Pasadena, CA.  October, 2006.

%\textit{The M51 Spectral Cube.} First Status Report to the Spitzer VGSF Committee.  Spitzer Science Center, Pasadena, CA. September, 2006. 
 
      
\section{PUBLICATIONS, CONFERENCE PROCEEDINGS, AND PRESS RELEASES}

\textit{Mapping the Spatial Distribution of Warm H$_2$ in Nearby Galaxies with the Spitzer Infrared Spectrograph.}  {\bf Brunner, G.}, Dufour, R., Sheth, K., Armus, L., Wolfire, M., Vogel, S., and Schinnerer, E.  2009.  \underline{The Evolving ISM in the Milky Way and Nearby Galaxies.}

\textit{Observations of M33 H II Regions: The Ne/S Ratio, Metallicity, and Ionization Variations.}  Rubin, R.H., Simpson, J.P., McNabb, I.A., {\bf Brunner, G.}, Colgan, S.W.J., Dufour, R., Pauldrach, A.W.A., Browne, A.D., Zhang, R., and Csongradi, E.J.  2009.  \underline{The Evolving ISM in the Milky  
Way and} \underline{Nearby Galaxies.}

\textit{Warm Molecular Gas in M51: Mapping the Excitation Temperature and Mass of H$_2$ with the Spitzer Infrared Spectrograph.} {\bf Brunner, G.}, Sheth, K. Armus, L., Wolfire, M., Vogel, S.N., Helou, G., Dufour, R.J., Smith, J.D.T., and Dale, D.A.  2008.  \underline{The Astrophysical Journal.} 675, 316.

\textit{Spitzer Finds Cosmic Neon's Sweet Spot.}  Rubin, R., Simpson, J., Colgan, S., McNabb, I., Erickson, E., Haas, M., Citron, R., Dufour, D., {\bf Brunner, G.}, Pauldrach, A.  16 May 2008.\\ http://www.spitzer.caltech.edu/Media/happenings/20080516/

\textit{Spitzer Observations of M33 and the Hot Star, HII Region Connection.} Rubin, R.H., Simpson, J. P., Colgan, S. W. J., Dufour, R. J., {\bf Brunner, G.}, Ray, K. L., Erickson, E. F., Haas, M. R., Pauldrach, A. W. A., and Citron, R. I.  2008.  \underline{The Monthly Notices of the Royal Astronomical Society.} 387, 45.


\section{REFERENCES}
Available upon request
 
%\section{EXTRACURRICULAR ACTIVITIES}          
    %Epsilon Delta Sigma Public Relations Committee \\         
    %Rensselaer Ski Club     \\     
    %Bergen County Task Force Student Liaison 1986  \\        
    %LEADD (Legislators and Educators Against Drunk Driving) Chairman
     %1985-86  \\        
    %Youth Group - Temple Beth Or Activities Chairman 1985-86          
 
\end{resume}
\end{document}